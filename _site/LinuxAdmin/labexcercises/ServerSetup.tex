\documentclass[12pt]{article}

\usepackage{fontawesome}
\usepackage{hyperref}

\hypersetup{
    colorlinks=false,
    pdfborder={0 0 0},
}


\title{Server Setup}
\author{
        Adrianna Holden-Gouveia \\
        Website: \url{https://aholdengouveia.name}\\ 
        \date{\vspace{-5ex}}
        %Email: \href{mailto:admin@aholdengouveia.name}{admin@aholdengouveia.name} \\
        \faLinkedin{: aholdengouveia} \\
        \faGithub {: aholdengouveia} \\
        %\faTwitter {: aholdengouveia} \\
        }

%S\date{\today}


\begin{document}    

\maketitle

%\begin{abstract}
%This is a template for Linux Administration Lab work
%\end{abstract}
%\tableofcontents

%\section*{Objectives:}
%\begin{enumerate}
 %   \item Objective 1
  %  \item Objective 2
%\end{enumerate}
\section*{Welcome to Acme Corporation!}
\subsection*{You have accepted the post of Systems Administrator for Acme corporation.} 
\subsection*{Congratulations on your new job!}

If you've never done technical documentation before, go through the module on Technical documentation and portfolios before attempting this lab.

    \begin{itemize}
\item References for an Introduction to Servers, a video and some notes are available at my website

https://www.aholdengouveia.name/LinuxAdmin/IntroServers.html 


\item I also have a section on how to do documentation 

https://www.aholdengouveia.name/LinuxAdmin/Documentation.html


\end{itemize}

\section*{Complete the following problems}

\subsection*{We need two Linux systems that are servers.  Virtual machines are the recommended option}
    \begin{enumerate}
        \item You will need to use 2 different Linux Distributions (also known as distros or flavours), one for each setup.
        \item One must be CentOS, the other can be another server distro of your choice such as Ubuntu Server, Debian, OpenSUSE or Fedora.  Ubuntu Server or Debian is strongly recommended
        \item You MUST use the server version, desktop Operating Systems  are NOT ok for this! 
        \item Do not use the GUI or the minimal install. You can use a GUI to do the install, but once the server is booted it should be command prompt only.
        \item Make sure you turn on Internet access for both servers.  This needs to be tested
        \item Include a way to share files (such as SmarTTY or SCP)


    \end{enumerate}



\section*{Deliverables}

Target market is someone trying to follow along with you to make their own machines, assume some technical knowledge but not expertise. 

Screenshots are helpful to go with your descriptions.

\begin{itemize}
    \item A short document showing how you did installs.
    \item A short document saying what you have included in your installs, including any software you've added and any configurations you've changed. 
    \item Make sure to include documentation on how to use your your file share as well as how you setup your file share.
\end{itemize}
 
\end{document}