\documentclass[12pt]{article}

\usepackage{fontawesome}
\usepackage{hyperref}

\hypersetup{
    colorlinks=false,
    pdfborder={0 0 0},
}


\title{Lab Template}
\author{
        Adrianna Holden-Gouveia \\
        Website: \url{https://aholdengouveia.name}\\ 
        \date{\vspace{-5ex}}
        %Email: \href{mailto:admin@aholdengouveia.name}{admin@aholdengouveia.name} \\
        \faLinkedin{: aholdengouveia} \\
        \faGithub {: aholdengouveia} \\
       % \faTwitter {: aholdengouveia} \\
        }

%S\date{\today}


\begin{document}    

\maketitle

%\begin{abstract}
%This is a template for Linux Administration Lab work
%\end{abstract}
%\tableofcontents

\section*{Objectives:}
\begin{enumerate}
    \item Objective 1
    \item Objective 2
\end{enumerate}
\section*{Complete the following problems}

References, a video, a PowerPoint and some notes are available at my website
\url {https://www.aholdengouveia.name}

\subsection*{Please include the command, a screenshot showing it works as intended, cite all sources you used, and give a short explanation of how the command works and why.}
    \begin{enumerate}
        \item 1
        \item 2
        \item 3
        \item 4
        \item 


    \end{enumerate}



\section*{Deliverables}
A text document with the grep statements used to get each result.  Clearly labeled so I can see which answer goes to which question.


Please include the command, a screenshot that includes the command and a few results showing it works as intended, cite all sources you used, and give a short explanation of how the command works and why. Make sure everything is in your own words.  Think of the explanation as you trying to walk someone through how to use grep when they haven't heard of it before.
\end{document}