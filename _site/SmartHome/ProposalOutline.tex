\documentclass[12pt]{article}

\usepackage{fontawesome}
\usepackage{hyperref}

\hypersetup{
    colorlinks=false,
    pdfborder={0 0 0},
}


\title{Proposal Outline of a more Secure and Open Smart Home}
\author{
        Adrianna Holden-Gouveia \\
        Website: \url{https://aholdengouveia.name}\\ 
        Email: \href{mailto:admin@aholdengouveia.name}{admin@aholdengouveia.name} \\
        \faLinkedin{: aholdengouveia} \\
        \faGithub {: aholdengouveia} \\
        \faTwitter {: aholdengouveia} \\
        }

\date{\today}


\begin{document}    

\maketitle

\begin{abstract}
This outline is for researching Internet of Things (IoT) devices for a smart home. IoT devices in the past have been done by companies looking to make profits from both the sale of the device, and the data gathered from the people using the device.  Open Source is the concept that rather then having a single person or company in charge of standards and implementation, it's done by a community anyone can be a part of.  But putting Open Source devices in our homes we can control what data if any is being collected, and at the end we have complete control over our data and devices.
\end{abstract}
%\tableofcontents


\section*{Outline}

\begin{enumerate}
    \item What is the Internet of Things (IoT)
    \begin{enumerate}
        \item Example: Lights
        \item Example: Magic Mirrors
        \item Example: Cameras
        \item Example: Environmental monitoring
    \end{enumerate}
    \item Appeal of home items 
    \item Open Source
    \begin{enumerate}
        \item Version control
        \item Accessible and shareable
        \item Cost of DIY vs branded items
    \end{enumerate}
    \item Data security and privacy
    \begin{enumerate}
        \item Data collection
        \item Uses of data and limits on what can be collected Both USA and worldwide
        \item How open source DIY IoT helps
        \item IoT botnet risks of unsecured devices
    \end{enumerate}
    \item Why you should consider only using Open Source personally controlled IoT devices
\end{enumerate}

\section*{Summary}
One of the tougher issues in technology can be making it feel accessible to a wider audience.  By creating easy to read documentation people will be able to see what to do, and how the technology can make their lives better.  By making it open source they can change it if they want, share freely, and the community can be a part of where the project goes next. Taking away the extra costs of premade equipment, and branded items we are able to save enough money to make this accessible to almost anyone. Putting everything on version control allows everyone to see what has been done, changes, and even easily participate themselves. 

\end{document}