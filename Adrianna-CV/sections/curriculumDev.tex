
\section{Curriculum Development}




\denseouterlist{

\entrymid[\textbullet]
{\textbf{Introduction to to IT}}{CTN 110}
{This is a gateway course to overview core aspects of Information Technology including hardware, networking, applications and security.  This is a breadth course to help students get the fundamentals of IT and how it relates to things they might be interested in pursuing, jobs that are out there, as well as getting the fundamental concepts understood so students are prepared for later classes.}

\entrymid[\textbullet]
{\textbf{Data Management}}{CIS113}
{This is an introductory course in data literacy and databases.  The course includes aspects of database design including entity relationship modeling, tables, and queries.  Students will utilize database applications and Structured Query Language (SQL). Data science including data collection, modeling, and visualization will be discussed. Best practices for data security and privacy in regards to current regulations around the world will be emphasized in this course.These fundamental concepts are introduced and reinforced with practical labs that can be customized based on student interest.}

\entrymid[\textbullet]
{\textbf{Information security}}{CIS115}
{This course is required for all technical majors.  This is hands-on style labs focused on the fundamental principles for information security, what everyone needs to know about cyber security in this day and age. There are minimal pre-requisites for this course so that it can be accessible to everyone.  Topics and labs include how to identify and remove spyware, viruses and all types of malware.  The importance of general cybersecurity knowledge as it relates to everyday life including password management best practices, social engineering prevention and education, and how basic technology we take for granted works and is secured.  Things like the Internet of Things (IoT) and how that is secured as it relates to cars, smart home devices and security cameras, but also how it relates to the laws being passed around the world such as General Data Protection Regulation (GDPR) and how that is affecting businesses small and large.}

\entrymid[\textbullet]
{\textbf{Introduction to Linux}}{CIS117\footnotemark}
{This course is designed as an introduction to the basics of Linux for all technical majors.  This course utilizes hands-on style labs to explore basic system navigation, file manipulation, text processing and shell scripting.  I designed the course to use Bash primarily. }

\entrymid[\textbullet]
{\textbf{Introduction to Operating Systems}}{CIS121}
{This course is not currently being offered.  It was designed to be a part Linux part sub-systems course. There were hands-on style labs, as well as a deep dive into the pieces of the kernel, and how computers are put together from the software to how the software interacts with the hardware. }

\entrymid[\textbullet]
{\textbf{Programming for IT}}{CIS153\footnotemark[\value{footnote}]}
{I designed this course to be the programming course that IT and Networking Security majors will take, this course is currently done in Python.  We look at not just the basic syntax of data types, control flows and GUI applications, but how problem solving skills can be used with programming and scripting to assist in your job.  There is also some project management and Agile techniques incorporated into the hands-on style assignments to work on how to break up problems and plan out your projects into milestones and goals.}

\entrymid[\textbullet]
{\textbf{Advanced Computer Security}}{CIS215}
{This course is a continuation of our Information Security course.  This course is a deeper more advanced look at all topics in cyber security.  We cover cryptography, access control mechanisms, IDS/IPS, vulnerability analysis, networking intrusion and security auditing.  We focus on how information security is handled in a corporate environment.}

\entrymid[\textbullet]
{\textbf{Advanced Data Management}}{CIS210}
{This course is a continuation of our introduction to data course.  This course is a deeper more advanced look at all topics regarding data and databases.  This course will focus on advanced techniques and concepts in database management. Emphasis will be on Structured Query Language (SQL)  but will also include some NoSQL, and current industry best practices for utilizing data and databases. This course will be using hands-on exercises and real-world case studies for topics such as database optimization, database scalability, performance tuning, and concurrency control.}

\entrymid[\textbullet]
{\textbf{Linux Administration}}{CIS245\footnotemark[\value{footnote}]}
{I created as a capstone style course for the IT and Networking Security degrees.  This course focuses on what system administrators need to know on the job, both in terms of tools, but also topics that aren’t usually covered in academia such as documentation, the importance of keeping your accomplishments listed so you're prepared for remote work and quarterly or yearly reviews. We also cover how to integrate previous knowledge such as Python and Regular Expressions (Regex) into streamlining your work and seeing where they can go next based on individual interests, as well as strengthening troubleshooting skills. This course utilizes hands-on labs for topics such as software configuration and installation, while employing best practices from the ground up for all aspects of systems administration. }


%\entrymid[\textbullet]
%{\textbf{1cr Job Skills Course}}{COP109}
%{Currently creating a 1 credit course for job skills without an internship for those unable to take the internship class for whatever reason.  This course is done in close collaboration with the career center to ensure it meets current best practices.}

\entrymid[\textbullet]
{\textbf{Internship/Co-op}}{COP110}
{Created to follow current best practices for internships and cooperative education, focused on STEM students. This course is designed to prepare Students in the CIS programs, working with their internship site to prepare them for jobs in the future, with needed skills such as resume building, portfolios, interviews both technical and HR style, as well as professional dress, social media and other topics that are helpful in the work place.}
}

\footnotetext[1]{This indicates it was a brand new course for the college, not a redesign of an older course}

