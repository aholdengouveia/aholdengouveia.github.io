\documentclass[12pt]{article}

\usepackage{fontawesome}
\usepackage{hyperref}

\hypersetup{
    colorlinks=false,
    pdfborder={0 0 0},
}


\title{Networking}
\author{
        Adrianna Holden-Gouveia \\
        Website: \url{https://aholdengouveia.name}\\ 
        \date{\vspace{-5ex}}
        %Email: \href{mailto:admin@aholdengouveia.name}{admin@aholdengouveia.name} \\
        \faLinkedin{: aholdengouveia} \\
        \faGithub {: aholdengouveia} \\
        %\faTwitter {: aholdengouveia} \\
        }

%S\date{\today}


\begin{document}    

\maketitle

%\begin{abstract}
%This is a template for Linux Administration Lab work
%\end{abstract}
%\tableofcontents

%\section*{Objectives:}
%\begin{enumerate}
%    \item Objective 1
%    \item Objective 2
%\end{enumerate}
\section*{Complete the following problems}

References, a video, a PowerPoint and some notes are available at my website
https://www.aholdengouveia.name/LinuxAdmin/networking.html



\subsection*{Go through both servers and complete the following problems.  Make sure your deliverables include documents for BOTH servers clearly labeled with what the document is and which server it's for}
    \begin{enumerate}
        \item Document network setup. Your setup should include how to find all the networking information, as well as the basics on how your server does networking by default. This should include locations of any files you've created or need to reference. You should include any programs you've installed or updated as well, with dates of when you've done those things. Include a short document including how you did any of the installs or updates. 
        \item Write a script to dump network info (including ip, dns, open ports and anything else you think is useful) to the commandline and to a file.  Make sure the name of the file includes the date and the purpose.  This needs to be more then just a single command or two, really think about the format of the file it's going into, what's useful, and what you would want if someone handed you this file. 
    \end{enumerate}

\section*{Deliverables}
Target Market is someone trying to take care of your server that hasn't used them before, assume technical knowledge but not expertise.  Screenshots are helpful to go with your descriptions. 

\begin{itemize}
    \item A  document that is your current network setup, including notes of what is installed, the last time it was updated and any changes you've made to the defaults. Make sure to include a short paragraph explaining how networking works on your server by default as well as how to find all relevant information/scripts/configs/tools.
    \item At least one page of research on what you've chosen to add to your script and why.  Remember to cite your sources and explain why you think this additions are valuable.Sources do not count towards the one page.
    \item A well commented and well tested script, and a sample run of your script on each server.  Include any instructions needed for running the script.  Such as, does it need to be run in a special folder? does it save the textfile to a special place (it should) and where that might be? Which commands you've chosen to put in the script and why.
    \item link to your GitHub where your script has been uploaded
\end{itemize}


\end{document}